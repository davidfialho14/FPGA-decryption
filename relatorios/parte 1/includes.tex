\usepackage[portuguese]{babel}
\usepackage[utf8]{inputenc}
\usepackage{textgreek}
\usepackage{graphicx}
\usepackage{amsmath}
\usepackage{float}
\usepackage{listings}
\usepackage{color}
\usepackage[]{algorithm2e}
\usepackage{algorithmic}
\usepackage{minted}

\definecolor{keywordcolor}{rgb}{0,0.4,0.7}
\definecolor{commentcolor}{rgb}{0.4,0.4,0.4}
\definecolor{mygray}{rgb}{0.5,0.5,0.5} % line counter color
\definecolor{mymauve}{rgb}{0.55,0.28,0.62} % string color
\definecolor{codebackground}{rgb}{0.95,0.95,0.95}

\lstset{ %
	backgroundcolor=\color{codebackground}, % choose the background color; you must add \usepackage{color} or \usepackage{xcolor}
	basicstyle=\ttfamily \footnotesize, % the size of the fonts that are used for the code
	breakatwhitespace=false, % sets if automatic breaks should only happen at whitespace
	breaklines=true, % sets automatic line breaking
	captionpos=b, % sets the caption-position to bottom
	commentstyle=\color{commentcolor}, % comment style
	deletekeywords={...}, % if you want to delete keywords from the given language
	escapeinside={\%*}{*)}, % if you want to add LaTeX within your code
	extendedchars=true, % lets you use non-ASCII characters; for 8-bits encodings only, does not work with UTF-8
	keepspaces=true, % keeps spaces in text, useful for keeping indentation of code (possibly needs columns=flexible)
	keywordstyle=\color{keywordcolor}, % keyword style
	numbers=left, % where to put the line-numbers; possible values are (none, left, right)
	numbersep=5pt, % how far the line-numbers are from the code
	numberstyle=\tiny\color{mygray}, % the style that is used for the line-numbers
	rulecolor=\color{black}, % if not set, the frame-color may be changed on line-breaks within not-black text (e.g. comments (green here))
	showspaces=false, % show spaces everywhere adding particular underscores; it overrides 'showstringspaces'
	showstringspaces=false, % underline spaces within strings only
	showtabs=false, % show tabs within strings adding particular underscores
	stepnumber=1, % the step between two line-numbers. If it's 1, each line will be numbered
	stringstyle=\color{mymauve}, % string literal style
	tabsize=2 % sets default tabsize to 2 spaces
}
\renewcommand{\lstlistingname}{Código} % Listing x -> Código x
\def\lstlistingautorefname{Código}